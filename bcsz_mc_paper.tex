\documentclass{mc2015}

%%%%%%%%%%%%%%%%%%%%%%%%%%%%%%%%%%%%%%%%%%%%%%%%%%%%%%%%%%%%%%%%%%%%%
\usepackage[T1]{fontenc}         % Use T1 encoding instead of OT1
\usepackage[utf8]{inputenc}      % Use UTF8 input encoding
\usepackage{microtype}           % Improve typography
\usepackage{booktabs}            % Publication quality tables
\usepackage{amsmath}
\usepackage{graphicx}
\usepackage{float}
\usepackage[exponent-product=\cdot]{siunitx}
\usepackage[colorlinks,breaklinks]{hyperref}
\hypersetup{linkcolor=black, citecolor=black, urlcolor=black}

\usepackage{lipsum}

\def\equationautorefname{Eq.}
\def\figureautorefname{Fig.}

%%%%%%%%%%%%%%%%%%%%%%%%%%%%%%%%%%%%%%%%%%%%%%%%%%%%%%%%%%%%%%%%%%%%%
% Insert authors' names and short version of title in lines below

\authorHead{Maginot, Ragusa, and Morel}
\shortTitle{BCSZ}

%%%%%%%%%%%%%%%%%%%%%%%%%%%%%%%%%%%%%%%%%%%%%%%%%%%%%%%%%%%%%%%%%%%%%
\begin{document}

\title{A Non-negative, Non-linear, Petrov-Galerkin Method for Bilinear Discontinuous differencing of the $S_N$ Equations}

\author{Peter G. Maginot}
\author{Jean C. Ragusa\footnote{Corresponding author} }
\author{Jim E. Morel}
\affil{Department of Nuclear Engineering\\
Texas A\&M University \\
  3133 TAMU, College Station, TX 77843 \\
 pmaginot@tamu.edu; jean.ragusa@tamu.edu; morel@tamu.edu}

\maketitle

\begin{abstract}
We have developed a new, non-negative, non-linear, Petrov-Galerkin discontinuous finite element method (NNPGDFEM), for use in conjunction with Galerkin bilinear discontinuous  
(BLD) finite element differencing of the 2-D Cartesian geometry $S_N$ equations for quadrilaterals on an unstructured mesh.  
This work is an extension of the idea that drove our previous development of a NNPGDFEM for use with linear discontinuous (LD) differencing of the 2-D Cartesian geometry $S_N$ equations for rectangular mesh cells.
We present the theory and equations that describe the new method.  
Additionally, we numerically compare the accuracy of our proposed method to the accuracy of the BLD scheme (without lumping) and the subcell corner balance method (equivalent to a ``fully'' lumped bilinear discontinuous scheme) for a test problem in which the BLD solution contains negative angular flux solutions

\emph{Key Words}: Radiation transport, DFEM, non-negative, bilinear, quadrilaterals
\end{abstract}

%% Final PDF file size should be no more than 4 MB.  Recommended paper length is 10-12 pages.  %%

%%%%%%%%%%%%%%%%%%%%%%%%%%%%%%%%%%%%%%%%%%%%%%%%%%%%%%%%%%%%%%%%%%%%%
\section{Introduction}

%
% - Negativities are a problem in DFEM radtran
% - Need bilinear dfem for thick diffusion limit (cite Adams paper)
% - Solution method like CSZ paper, not Don's paper
%
%

Discontinuous finite element discretizations of the $S_N$ neutron transport equation can result in negative angular flux solutions.  
These negative angular flux solutions are non-physical, but inherent to the mathematics that define the spatial differencing scheme.
Several researchers have examined a the different methods (matrix lumping, fix-ups, and strictly non-negative solution representations) that inhibit or eliminate the negativities of the linear discontinuous (LD) finite element scheme on  a variety of spatial mesh cell types (slab, rectangular, triangular, \dots), for the $S_N$ neutron transport equation.
However, Adams showed that  a bilinear discontinuous (BLD) scheme (mapped to a reference element) is required  to preserve the thick diffusion limit of the 2-D $S_N$ equations on quadrilaterals.
The neutron transport analogs of radiative transfer spatial discretizations that preserve the equilibrium diffusion limit must preserve the thick diffusion limit.
Thus, we are interested in finding accurate and robust BLD discretizations of the neutron transport equation as a first step towards more accurate radiative transfer simulations.

To our knowledge, only matrix lumping has been considered  in an attempt to inhibit negative angular flux solutions of the (BLD) finite element spatial discretization.
The more common forms of matrix lumping, mass matrix and mass matrix with surface matrix lumping, inhibit negative angular flux solutions, but do not guarantee a ``robust''  BLD angular flux solution.
A spatial discretization is defined as being robust if the discretization yields strictly non-negative angular flux solutions, regardless of spatial mesh cell optical thickness and discrete ordinates direction.
The fully lumped BLD (FLBLD) scheme, that is equivalent to the subcell balance method on rectangles, is the only BLD scheme that becomes robust with matrix lumping.
Though robust, the FLBLD scheme has been shown to be less accurate than the unlumped BLD (UBLD) scheme for spatial mesh cells of thin and intermediate thicknesses.


References can be typeset properly using the provided \textsc{Bib}\TeX style
file. See examples of a journal~\cite{journal}, conference
proceedings~\cite{proceedings}, book~\cite{book}, and
miscellaneous~\cite{misc}.

References to websites are discouraged, but acceptable if absolutely necessary.  It is the author?s responsibility to check links in the pdf file.


\subsection{Second or Subsequent Major Heading}

A logical division of your paper into sections makes it much easier to understand.    

\subsection{Subsection Title: First Character of Each Non-trivial Word is Uppercase}

Equations (Equation \ref{eqn:sample}) should be centered and sequentially numbered to the flush right of the formula.

\begin{equation}
  1+1=2
  \label{eqn:sample}
\end{equation}

\noindent
The continuation of a paragraph after an equation is not indented

\subsubsection{Sub-subsection level and lower: only first character uppercase}

Figures and tables should appear as closely as possible to where they are first cited, e.g. 
Fig. \ref{fig:sample}, in the text.  Figures are numbered in Arabic numerals, with the caption centered below the figure, in boldface. 

\begin{figure}[H]
  \centering
  %\includegraphics[width=3in]{figure.png}
  \caption{Sample Figure. Color is permitted, but must be readable if printed.}
  \label{fig:sample}
\end{figure}

When importing figures or any graphical image please verify two things:
\begin{enumerate}
\item Any number, text or symbol is no smaller than 10-point after reduction to the actual window in your paper;
\item That it can be translated into PDF.
\end{enumerate}

Tables, like Table \ref{tab:sample}, are numbered in Roman numerals, with the caption centered above the table, in \textbf{boldface}.  
Double-space before and after the table.

\begin{table}
  \centering
  \caption{Sample table: accuracy of nodal and characteristic methods}
  \begin{tabular}{lcccc}
    \toprule
    Mesh & 8 x 8 & 16 x 16 & 32 x 32 & 64 x 64 \\
    \midrule
    Nodal & \num{1.000e-1} & \num{2.500e-2} & \num{6.250e-3} & \num{1.563e-3} \\
    Characteristic & \num{1.000e-1} & \num{2.500e-2} & \num{6.250e-3} & \num{1.563e-3} \\
    \bottomrule
  \end{tabular}
  \label{tab:sample}
\end{table}

%%%%%%%%%%%%%%%%%%%%%%%%%%%%%%%%%%%%%%%%%%%%%%%%%%%%%%%%%%%%%%%%%%%%%
\section{Conclusions}

Present your summary and conclusions here.

%%%%%%%%%%%%%%%%%%%%%%%%%%%%%%%%%%%%%%%%%%%%%%%%%%%%%%%%%%%%%%%%%%%%%
\section{Acknowledgments}

Acknowledge the help of colleagues and sources of funding, as appropriate, including Paul Romano and Tom Sutton, 
who provided this template that is similar to a template from past M\&C meetings.

%%%%%%%%%%%%%%%%%%%%%%%%%%%%%%%%%%%%%%%%%%%%%%%%%%%%%%%%%%%%%%%%%%%%%
\setlength{\baselineskip}{12pt}

\bibliographystyle{mc2015}
\bibliography{references}

%%%%%%%%%%%%%%%%%%%%%%%%%%%%%%%%%%%%%%%%%%%%%%%%%%%%%%%%%%%%%%%%%%%%%
\appendix
\section{}

If necessary, include Appendices numbered in upper case alphabetical order.

In order to ensure a uniform, professional look to the proceedings, please
\emph{\textbf{do not modify}} the format of this template without checking with
the organizers first.

\end{document}
